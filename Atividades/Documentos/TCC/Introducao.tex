A evolução dos sistemas de comunicação móveis, impulsionada pela crescente demanda por comunicações mais rápidas e eficientes, tem levado à implementação de uma variedade de serviços, incluindo aplicações multimídia, desenvolvimento web e aplicações IoT \cite{John2016}. No entanto, essa evolução também trouxe desafios significativos, como a necessidade de melhorar a eficiência energética, tanto para dispositivos móveis, visando aumentar a autonomia da bateria, quanto para estações de rádio base, visando reduzir o consumo de energia devido às perdas de calor. Para atender a essas demandas, estratégias de modulação que alteram tanto a fase quanto a amplitude de ondas portadoras em radiofrequência se tornaram essenciais \cite{Kenington2000}. Além disso, a modulação na amplitude requer linearidade na transmissão para evitar erros e interferências na comunicação entre usuários vizinhos \cite{Cripps2006}. Essa complexa tarefa recai sobre o projetista do PARF (Amplificador de Potência de Rádio Frequência), que enfrenta o desafio de desenvolver um hardware eficiente em termos energéticos e linear ao mesmo tempo, uma vez que esses dois objetivos podem entrar em conflito \cite{Chavez2018}. Uma solução para contornar esse desafio é a implementação de um pré-distorcedor de Sinais Digital em Banda Base, que visa compensar a distorção causada pelo PARF \cite{Cripps2006}. O DPD (Pre-distorcedor Digital) é conectado em cascata ao PARF e requer um modelo de alta precisão e baixa complexidade computacional para representar as características de transferência direta e inversa do PARF. Existem duas abordagens para modelar o PARF: modelos físicos, que são detalhadas e computacionalmente complexos, e modelos empíricos, que se baseiam em medições de entrada e saída do PARF, com menor complexidade computacional, mas com uma possível diminuição da precisão. Devido às exigências rigorosas de frequência de operação, a paralelização das operações torna-se essencial, e as FPGAs (Matriz de Portas Programáveis em Campo) emergem como uma alternativa viável para a implementação de circuitos pré-distorcedores \cite{Pedroni2010}. As FPGAs são dispositivos lógicos programáveis que permitem a reconfiguração física de componentes de eletrônica digital, acelerando processos e suportando operações paralelas e sequenciais. Neste contexto esse projeto foi planejado com os seguintes objetivos geral e específicos:

\section{Objetivo Geral}
O desenvolvimento de um circuito integrado dedicado para um pré-distorcedor digital na tecnologia BiCMOS 130 nm 8HP.
\section{Objetivos Específicos}

Para alcançar o objetivo geral, este trabalho foi desenvolvido com base nos seguintes objetivos específicos:

\begin{enumerate}
    \item Modelar com precisão o PA em software;
    \item Modelar o DPD em software a partir da modelagem do PA;
    \item Implementar o DPD em hardware utilizando uma HDL;
    \item Desenvolver o design do circuito integrado.
\end{enumerate}

