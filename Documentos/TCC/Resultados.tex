Conforme dito no capitulo \ref{chap:mati} o trabalho foi divido em 4 etapas, nas quais a primeira etapa consistiu no estudo dos DPDs e nos métodos de modelagem deles, a segunda etapa consistiu na implementação desta modelagem em software, que foi optado em utilizar o python, a etapa 3 que consiste na implementação do modelo de DPD escolhido em hardware ultilizando a linguagem VHDL e finalmente na quarta etapa e feita o design do circuito integrado do circuito integrado.

o grafico presente na figura \ref{fig:bits}, onde observa-se que a partir de 7 bits de resolução não tem uma melhora expressiva no NMSE do sinal.

\begin{figure}[ht!]
    \centering
    \captionsetup{justification=centering}
    \caption*{Fonte: Autor}
    \includegraphics[width=0.5\textwidth]{bits.png}
    \caption{Grafico Numero de bits x NMSE}
    \label{fig:bits}
\end{figure}
