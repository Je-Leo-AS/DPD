\chapter*{\normalsize ANEXO A - função que calcula matriz de confusão em python}
\small % Altera o tamanho da fonte
\begin{lstlisting}[language = Python, label={cod:mp}]
def mp(P, M, xn):
    L = xn.shape
    XX = np.zeros((L[0] - M, P * (M+1)), dtype=np.complex128)
    for l in range(M+1, L[0]):
        for p in range(1, P+1):
            for m in range(0, M+1):
                XX[l-M-1, ((p-1)*(M+1))+m] = (np.abs(xn[l-m])**(2*p-2)*(xn[l-m]))[0]
    return XX

XX_ext = mp(P, M, in_data_ext)
coefficients, _, _, _ = np.linalg.lstsq(XX_ext, out_data_ext[M:], rcond=None)
predicted_val = XX_val @ coefficients

\end{lstlisting}
Anexo: texto ou documento não elaborado pelo autor, que serve de fundamentação, comprovação e ilustração.

\chapter*{\normalsize ANEXO B - Digite o cabeçalho do anexo}