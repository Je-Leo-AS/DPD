A evolução dos sistemas de comunicação sem fio acarretou na implementação de diversas aplicações moveis e sem fio como desenvolvimento web, aplicação IoT, entre outros. Neste cenário, melhorar a eficiência energética se torna uma alternativa desejável tanto para os dispositivos móveis que buscam melhorar a autonomia das suas baterias, quanto para as estações de rádio base, que buscam reduzir sues desperdício em perdas de calor. No entanto uma melhor eficiência energética implica em uma menor linearidade nos sistemas de amplificação de sinais, presentes nos sistemas transmissores de sinais de rádio. Isto é importante de ser ressaltado, pois a banda reservada para aplicações móveis é reduzida, de forma que para se alcançar maiores taxas de transmissão é necessário alternar estratégias de modulação tanto da fase, quanto da amplitude da onda portadora. E isso conflitoso já que a modulação AM é sensível a linearidade de forma que quanto mais linear um sistema menor erros de transmissão ocorrem. Sendo assim uma alternativa para contornar esse obstáculo, que é implementar um sistema, eficiente energeticamente e linear é a implementação de um Pré-Distorcedor Digital em cascata com um Amplificador de Potência. Portanto, o objetivo deste trabalho de conclusão de curso é o design de um circuito integrado dedicado de um DPD.  

\textbf{Palavras-chave}: VHDL, FPGA, DPD 