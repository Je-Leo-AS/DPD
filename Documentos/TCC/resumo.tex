    A evolução dos sistemas de comunicação sem fio acarretou na implementação de diversas aplicações moveis e sem fio como desenvovimento web, aplcação IoT, entre outros. Neste cenario, melhorar a eficiência energética se torna uma alternativa desejavel tanto para os dispositivos móveis que buscam melhorar a autonomia das suas baterias, quanto para as estações de rádio base, que buscam reduzir sues desperdicio em perdas de calor. No entanto uma melhor eficiência energética implica em uma menor linearidade nos sistemas de amplificação de sinais, presentes nos sistemas trasmissores de sinais de radio. Isto é importante de ser ressaltado, pois a banda reservada para aplicações móveis é reduzida, de forma que para se alcançar maiores taxas de trasmissão é necessário alternar estratégias de modulação tanto da fase, quanto da amplitude da onda portadora. E isso conflitoso ja que a modulação AM  é sensivel a linearidadede, de forma que quanto mais linear um sistema menores erros de transmissão ocorrem. Sendo assim uma alternativa para contornar esse obstaculo, que é implementar um sistemas, eficiente energéticamente e linear é a implementação de um Pré-Distorcedor Digital em cascata com um Amplificador de Potência. Portanto, o objetivo deste trabalho de conclusão de curso é o design de um circuito integrado dedicado de um DPD. 

\textbf{Palavras-chave}: VHDL, FPGA, DPD
