Conforme dito anteriormente o trabalho busca o desenvolvimento do design do circuito integrado de um DPD, a partir de um modelo validado em software, e em hardware, no caso, FPGA. Para isso esse projeto foi divido em 4 etapas principais:

\begin{itemize}
    \item Estudo DPD;
    \item Implementação em software;
    \item Implementação em FPGA;
    \item Design e validação do circuito integrado.
\end{itemize}

\section{Estudo dos DPDs}
A etapa consistiu no estudo dos DPDs, conforme apresentado no Capítulo \ref{chap:revi}, onde foi feito todo o levantamento sobre os tipos de modelagem dos DPDs. O objetivo deste estudo é entender as diferentes abordagens de modelagem, avaliar seus desempenhos e identificar as mais adequadas para a aplicação em amplificadores de potência.

\section{Implementação em software}

Nesta etapa, foi realizada a implementação do modelo DPD em software, utilizando a linguagem de programação Python. Python é uma linguagem amigável, amplamente difundida na comunidade acadêmica e prática de usar.

Para essa modelagem, foram coletados sinais de entrada e saída de um amplificador de potência classe AB, que utiliza um HEMT fabricado com tecnologia GaN. O amplificador foi excitado por um sinal portador de frequência de 900 MHz, modulado por um sinal de envelope WCDMA 3GPP com aproximadamente 3,84 MHz de largura de banda. Os dados de entrada e saída do amplificador de potência foram medidos usando um VSA Rohde & Schwarz FSQ com uma taxa de amostragem de 61,44 MHz, conforme disponível em \cite{Bonfim2016}.

Em seguida, foi feito o cálculo de estimativa de sinal utilizando números em vírgula fixa. Para validar se a estimativa é fiel ao sinal original, foi utilizado o cálculo do NMSE. O NMSE alcançado para um polinômio de 2° grau com uma amostra memorizada foi de -23,57 dB. Em seguida foi feito 

% //TODO Adicionar arquivo foto da resolução
\section{Implementação em FPGA}


\section{Design e validação}
Finalmente, na última etapa, realiza-se o processo de concepção do circuito integrado do DPD como um circuito dedicado integrado na tecnologia BiCMOS 130 nm 8HP, utilizando as ferramentas do Cadence.
O fluxo de projeto VLSI  para design de um circuito integrado de aplicação específica, inclui a descrição do circuito em VHDL, síntese lógica utilizando as células padrão da tecnologia, place and route e simulações comportamentais e temporais. O diagrama do fluxo VLSI pode ser ilustrado pela figura \ref{fig:CMOS2010}.

\begin{figure}[ht!]
    \centering
    \captionsetup{justification=centering}
    \caption*{Fonte: \cite{CMOS2010}}
    \includegraphics[width=0.5\textwidth]{fluxovlsi.png}
    \caption{Fluxo de projeto VLSI}
    \label{fig:CMOS2010}
\end{figure}

No processo de desenvolvimento do circuito, várias etapas são executadas. Primeiro, há a simulação comportamental para verificar se o circuito VHDL atende às expectativas, utilizando um testbench em VHDL e a ferramenta Cadence NCLaunch. Em seguida, ocorre a síntese lógica, onde a partir do modelo comportamental, utiliza-se a ferramenta Genus para criar um modelo RTL com células padrão de tecnologia específica, considerando restrições de área, frequência e consumo de energia. A síntese gera dois arquivos: um com componentes e conexões, em Verilog, e outro com informações de atraso no formato SDF. A simulação pós-síntese é realizada para validar o netlist gerado, usando o mesmo testbench da simulação comportamental. Em seguida, na etapa de PAR, o layout é criado posicionando as células e realizando as conexões entre essas células, utilizando a ferramenta Innovus. Por fim, na simulação pós-PAR, o circuito é simulado considerando as resistências e capacitâncias parasitas. Cada etapa é fundamental para garantir o correto funcionamento do circuito.