A evolução dos sistemas de comunicação sem fio tem promovido a implementação de diversos serviços móveis, tornando essencial que esses sistemas operem com a máxima eficiência. Nesse cenário, a implementação de um DPD em cascata com o PA surge como uma alternativa de baixo custo e interessante para melhorar o desempenho desses sistemas.

O objetivo deste trabalho de conclusão de curso é implementar em hardware um DPD baseado no modelo de Polinômio de Memória. Para isso, o projeto foi dividido em quatro etapas: estudo do DPD e da modelagem matemática, modelagem do DPD em software, implementação do DPD em hardware e, finalmente, design do circuito integrado. 

Atualmente, o projeto está na etapa 3, que corresponde à implementação em hardware, conforme o cronograma exibido na tabela \ref{tab:cronograma}. 

Conclui-se, portanto, que as atividades estão sendo realizadas dentro dos prazos estabelecidos e que o projeto está progredindo conforme o esperado.
